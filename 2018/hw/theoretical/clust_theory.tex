\documentclass{article}
\usepackage[T2A]{fontenc}
\usepackage[utf8]{inputenc}
\usepackage[english,russian]{babel}
\addto\captionsrussian{%
  \renewcommand{\figurename}{Fig.}%
}
\usepackage{amsfonts,amssymb}
\usepackage{amsmath}
\usepackage{multicol}
\usepackage{graphicx}
\usepackage{color}
\usepackage{listings}
\usepackage{caption}
\usepackage[colorlinks,urlcolor=blue]{hyperref}
\DeclareCaptionFont{white}{\color{white}} %% это сделает текст заголовка белым
%% код ниже нарисует серую рамочку вокруг заголовка кода.
\DeclareCaptionFormat{listing}{\colorbox{gray}{\parbox{\textwidth}{#1#2#3}}}
\captionsetup[lstlisting]{format=listing,labelfont=white,textfont=white}
\usepackage{geometry}
\geometry{left=3cm}% левое поле 
\geometry{right=2cm}% правое поле 
\geometry{top=1cm}% верхнее поле 
\geometry{bottom=2cm}% нижнее поле
%\Russian
\title{Theoretical task 10.}
\date{}

\def\ShowAnswers{0}
\newcommand{\answer}[1]{\ifnum\ShowAnswers=1 \textbf{\\Solution:\\} {#1} \fi}
\renewcommand{\vec}[1]{\boldsymbol{#1}}
\newcommand{\tr}[1]{\text{tr}{#1}}

\begin{document}
\maketitle
{\it Recommendations: all solutions should be short, mathematically strict (unless qualitative explanation is needed), precise with respect to the stated question and clearly written. Solutions may be submitted in any readable format, including images. \newline
Submission link: \href{https://www.dropbox.com/request/aCKnPmaT1swqbJzLBfye}{here} }

\begin{enumerate}
	\item How would the steps of K-Means algorithm change, if in minimization criterion the euclidean distance is replaced by Manhattan distance (L1 distance)? What about its computational complexity?
	\item 
	\begin{enumerate}
		\item 	Show that K-Means criterion and the following criterion are equivalent:
	\[
	Q(C) = \sum_{k} \sum_{x_i, x_j \in C_k} \frac{ x_i^\top x_j}{|C_k|},
	\]
	where $C_k$ is cluster with label $k$
		\item Given that result, what technique can be used in K-Means algorithm?
	\end{enumerate}

\end{enumerate}


\end{document}
